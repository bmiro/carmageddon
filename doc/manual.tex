\section{Manual d'usuari}

La practica està escrita en \emph{python}\footnotemark, per tant és necessari tenir
l'interpret insta\lgem at.
Es recomana l'ús de \emph{pypy}\footnotemark en lloc de l'interpret per defecte ja 
que aquest implementa un \emph{just-in-time complier} i accelera notablement l'execució
de la pràctica.

Per dur a terme l'execució s'ha de fer amb el fitxer \texttt{prog.py} si es executat
sense parametres mostrarà quins suporta i alguns exemples d'execució.

Essencialment te dues maneres de funcionar, una es executar un test dels demanats a l'enunciat,
l'altre es executar un hill climbing o simulated anealing amb determinats paràmetres.

\subsection{Execució d'un test}
Per executar un test simplement s'ha de passar com a paràmetre el número del test a executar i 
aquest s'executara i mostrarà per pantalla el resultat.

En cas que sigui un test amb diferents opcions per defecte s'executa amb la opció que dona
una millor execució. Per tal de canviar dites opcions s'ha d'editar el fitxer prog.py anant
a la capçalera de la funció i canviant el paràmetre per defecte.

\begin{minted}[fontsize=\small]{python}
def test4_TemporalEvolution(operatorSet="1", initDistrib="allOneFirst"):
   ...
\end{minted}

\subsection{Execució simple}
Si sols volem executar una una cerca simple tenim dues opcions, una es carregar un fitxer de configuració
guardat i l'altre espeficicar tots els paràmetres.

Els paràmetres per a fer una execució simple són:

\begin{minted}[fontsize=\small]{bash}
pypy prog.py N M inicialitzacio heuristic conjuntOperadors algorismeCerca

\small
\begin{center}
\begin{tabularx}{\textwidth}{llX}
\textbf{parametre}  & \textbf{Valors} & \textbf{Descripció}\\
\midrule
N  & valor enter & Numero d'usuaris del servei\\
M  & valor enter & NUmero d'usuaris no conductors\\
\midrule
inicialització & allOneFirst & REF\\
               & fullFirst   & REF\\
\midrule
heuristic      & km  & Minimitza el número de km recorreguts\\
	       & veh & Mínimitza el numero de km i el numero de conductors\\
\midrule
conjuntOperadors      & 1 & Moure passatgers i llevar conductors buits\\
		      & 2 & Intercanviar passatgers i llevar conductors\\
\midrule
algorismeCerca & hillClimbing & Executa un Hill Climbing\\
               & simulatedAnnealing & Executa Simulated Annealing
\end{tabularx}
\end{center}
\normalsize


Els paràmetres per fer una execució des de un fitxer són:

\begin{minted}[fontsize=\small]{bash}
pypy prog.py fitxer.cfg heursistic conjuntOperadors algorismeCerca
\end{minted}

On fitxer es el nom del \texttt{fitxer.cfg} de configuració i la resta de paràmetres són
iguals a l'apartat anterior.

\footnotetext{\url{http://python.org}}
\footnotetext{El lloc oficial és \url{http://pypy.org/} però es pot
trobar precompilat i empaquetat per Debian/Ubuntu a \url{https://launchpad.net/~pypy/+archive/ppa}}

