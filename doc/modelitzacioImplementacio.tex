\section{Modelització del problema}
La pràctica ha estat desenvolupada en \emph{Python} el qual te una estructura de dades molt optimitzada
i integrada al llenguatge que es el diccionari.

\subsection{Representació de les dades}
Per la representació de les dades es fan servir 3 classes: passatger, conductor i estat.

On estat es l'estat requerit per les llibreries del AIMA. Passatger i conductor han estat
definits per facilitar l'elaboració de la pràctica.

\subsubsection{Passatger}
Cada passatger ve representat amb un identificador i els punts d'origen i destí tots ells són
camps invariables durant l'execució.

Assosciat a aquesta classe simplement hi ha els mètodes per obtenir el valors de dits camps.

\subsubsection{Conductor}
El conductor és una classe que hereda del passatger, així doncs també té el camp que l'identifica
el seu origen i destí. A més també té una capacitat màxima.
Per altra banda el conductor també inclou una llista dels noms dels passatgers que transportarà
durant tota la seva ruta.

Per gestionar aquesta clase són necessaris diveros mètodes més complicats que simples \emph{settes} i \emph{gettes}.
En primer lloc hi ha una funció \textt{pickupPassenger} per recollir un passatger, aquesta afegeix el passatger
a la llista de passatgers a transportar i comprova que en cap moment de la ruta es superi la capacitat
(\texttt{carOverflow}) del vehicle.

Per tal de sabre quina és la ruta que fa el conductor hi ha la funció \texttt{getRoute} la qual
fa una ordenació dels punts clau (\texttt{checkpoints}) per on es passa i registra a cada punt si es
l'origen del condcutor, l'origen d'un passatger, el destí del passatger o el destí del condcutor. Així
doncs un ruta tindria l'aspecte seguent: 


TODO ruta


Associat a una ruta ens interessa sabre la distància que recorr, per això hi ha la funció \emph{getKm}
que torna tal valor.





\subsubsection{Ciutat}
La citutat en si no requereix representació, no te perque estar guardara en memòria ja que senzillament
es pot gardar a cada passatger i conductor el seu punt d'inici i de destí com dues tuples \emph{XY}.

\subsubsection{Estat}

\subsection{Estat inicial}


\subsection{Operadors de transformació}


\subsection{Heurístic}