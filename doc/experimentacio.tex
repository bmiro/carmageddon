\section{Experimentació}

\subsection{Se\lgem ecció dels operadors}
Per a realitzar l'experiment 1 ha fixat en nombre d'usuaris a 200 i el nombre de conductors a 100.
Posteriorment s'ha executat 10 cops amb cada un dels conjunts d'operadors.


Els resultats són ja donats les mitjes són:

\begin{center}
\begin{tabular}{l|cccc}
         & Temps & Conductors & Distància & Estats (aprox)\\
\hline
Conjunt 1 & 0:02:14 & 98 & 1009304 & 9950 \\
Conjunt 2 & 0:01:30 & 98 & 1113629 & 5200
\end{tabular}
\end{center}


El conjunt d'operadors 1 produeix millor resultat que el conjunt 2.
Suposam que aixo es degut a que l'operador genera mes estats (gairebé el doble) per a cada iteració i simplement a més estats per triar s'agafen més bons.

Es pot veure que el conjunt 2 es notablement més ràpid però no en la mateix proporció que els estats generats. Això es deu a que aquests estats
son més costosos, entre d'altres coses perque un intercanvi implica dues operacions mentre que un canvi només una, entre d'altres detalls d'ambdos operadors.


\subsection{Se\lgem ecció de la generació de l'estat inicial}
Per realitzar l'experiment 2 s'ha fixat el nombre d'usuaris i el nombre de conductors com a l'anterior experiment.                                                                                   
S'ha executat 10 cops amb cada estrategia de generació inicial.
                                                                                                                                                                                                     
Els resultats són:


\begin{center}
\begin{tabular}{l|ccc}
         & Temps & Conductors & Distància\\
\hline
\emph{all one first} & 0:01:51 & 98 & 974543  \\
\emph{full first} & 0:02:08 & 98 & 1030151 
\end{tabular}
\end{center}
                                                                                                                                                                                  
La estrategia de generació inicial \emph{all one first} produeix millors resultats tant en temps com distància,
aquest fet es deu a que te els passatgers més repartits en contraposicó al \emph{full first} que els te
condensants i tarda més en repartir-los de manera adequada. De fet no te gaire sentit que els conductors
viatgin buits a l'inici ja que hem descartat el conjunt d'operadors que tal vegada haguessin pogut
jugar un bon paper amb el \emph{full first}.

\subsection{Se\lgem ecció dels paràmetres del Simulated Anealing}
Per a determinar els paràmetres que obtenen un millor resultat per a l'execució de l'algorisme \emph{simulated annealing},
hem executat reiterades vegades, aproximadament 10, l'algorisme amb diferents parametres per a \texttt{lambda} i \texttt{k}, concretament s'ha fet amb
\texttt{[0.1,0.01,0.001,0.0001]} i \texttt{[1,5,25,125]} repectivament, fixant el nombre d'iteracions en 2000.

El millor resultat s'ha obtingut amb els parametres k = 5 i lambda = 0.01, com indica el seguent grafic (fig. \ref{test3-gr1}).

\begin{figure}[H]
\begin{center} 
 \includegraphics[width=0.6\textwidth]{figures/test3-gr1.jpg}
\label{test3-gr1}
\end{center}
\end{figure}

Fixats aquests dos paràmetres en els valors que han obtingut millor resultat, s'ha ajustat el nombre d'iteracions executant diverses 
vegades l'algorisme amb un nombre diferent d'iteracions. Es pot veure (fig. \ref{test3-gr2}) en el seguent gràfic el millor valor per al nombre d'iteracions.

\begin{figure}[H]
\begin{center}
 \includegraphics[width=0.6\textwidth]{figures/test3-gr1.jpg}
 \label{test3-gr2}
\end{center}
\end{figure}


Aleshores, podem comparar els resultats obtinguts amb simulated Annealing i Hill climbing (fig. \ref{test3-gr3} i \ref{test3-gr4}).


\begin{figure}[H]
\begin{center} 
 \includegraphics[width=0.6\textwidth]{figures/test3-gr3.jpg}
\label{test3-gr3}
\end{center}
\end{figure}


\begin{figure}[H]
\begin{center}
 \includegraphics[width=0.6\textwidth]{figures/test3-gr4.jpg}
 \label{test3-gr4}
\end{center}
\end{figure}

Es pot observar que el Simulated Annealing obte millors resultats que el Hill Climbing. Per contra, té un temps d'execució molt més elevat.


\subsection{Observació de l'evolució temporal del Hill Climbing}
Per a l'experiment 4 s'ha fixat el nombre d'usuaris i el nombre de conductors com a l'experiment inicial.
En aquest cas, s'utilitza el conjunt d'operadors 1, i l'algorisme de generacio d'estats inicials \emph{all one first}.
S'ha executat 10 vegades cada simulacio.

Es pot observar com el temps d'execucio segueix clarament un patró quadràtic ascendent en funció del nombre d'usuaris
del servei.




\subsection{Se\lgem ecció dels valors de ponderació del segon heurístic}

\subsection{Se\lgem ecció del valor de \emph{M} per millor inicialització del problema}