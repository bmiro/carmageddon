\section{Experimentació}

\subsection{Se\lgem ecció dels operadors}

\subsection{Se\lgem ecció de la generació de l'estat inicial}

\subsection{Se\lgem ecció dels paràmetres del Simulated Anealing}
Per a determinar els paràmetres que obtenen un millor resultat per a l'execució de l'algorisme \emph{simulated annealing},
hem executat reiterades vegades l'algorisme amb diferents parametres per a \texttt{lambda} i \texttt{k}, concretament s'ha fet amb
\texttt{[0.1,0.01,0.001,0.0001]} i \texttt{[1,5,25,125]} repectivament, fixant el nombre d'iteracions en 2000.

El millor resultat s'ha obtingut amb els parametres [p1] [p2], com indica el seguent grafic.

(GRAFIC)

Fixats aquests dos paràmetres en els valors que han obtingut millor resultat, s'ha ajustat el nombre d'iteracions executant diverses 
vegades l'algorisme amb un nombre diferent d'iteracions. Es pot veure en el seguent gràfic el millor valor per al nombre d'iteracions.



(GRAFIC)

Aleshores, podem comparar els resultats obtinguts amb simulated Annealing i Hill climbing.

Per una banda pot observar que el simulated Annealing obté un resultat menys costós en termes de nombre de quilometres de la solucio,
 pero amb un cost temporal mes elevat.

(GRAFIC)

(GRAFIC)


\subsection{Observació de l'evolució temporal del Hill Climbing}

\subsection{Se\lgem ecció dels valors de ponderació del segon heurístic}

\subsection{Se\lgem ecció del valor de \emph{M} per millor inicialització del problema}